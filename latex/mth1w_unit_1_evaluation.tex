\documentclass[12pt, letterpaper]{article}

% PACKAGES
\usepackage{amsmath, amssymb, amsthm}  % Math packages
\usepackage{geometry}                  % Page margins
\usepackage{fancyhdr}                  % Headers/footers
\usepackage{graphicx}                  % For images
\usepackage{hyperref}                  % Clickable links
\usepackage{xcolor}                    % Color options
\usepackage{tikz}                      % Diagrams (optional)

% PAGE FORMATTING
\geometry{margin=1in}
\setlength{\parindent}{0pt}
\setlength{\parskip}{0.5em}

% HEADER/FOOTER
\pagestyle{fancy}
\fancyhf{}
\rhead{Your Name}
\lhead{Math Assignment \#X}
\cfoot{\thepage}

% THEOREM ENVIRONMENTS
\newtheorem{problem}{Problem}
\newtheorem{lemma}{Lemma}
\newtheorem{proposition}{Proposition}
\newtheorem{theorem}{Theorem}
\newtheorem{corollary}{Corollary}
\newenvironment{solution}{\paragraph{\textbf{Solution:}}}{\hfill$\blacksquare$}

% CUSTOM COMMANDS
\newcommand{\R}{\mathbb{R}}
\newcommand{\N}{\mathbb{N}}
\newcommand{\Q}{\mathbb{Q}}
\newcommand{\C}{\mathbb{C}}
\newcommand{\Z}{\mathbb{Z}}

% DOCUMENT INFO
\title{\textbf{Math Assignment \#X}}
\author{Your Name}
\date{\today}
\newcommand{\course}{Math 101: Calculus I}
\newcommand{\instructor}{Dr. Jane Smith}

% DOCUMENT START
\begin{document}
\maketitle
\hrule
\vspace{0.5cm}
\textbf{Course:} \course \hfill \textbf{Due Date:} October 31, 2023 \\
\textbf{Instructor:} \instructor \hfill \textbf{Student ID:} 123456789
\vspace{1cm}

% PROBLEM 1
\begin{problem}
  Let $f(x) = x^3 - 3x + 1$. Find all critical points of $f$ and determine their nature.
\end{problem}

\begin{solution}
  First, compute the derivative:
  \[
    f'(x) = 3x^2 - 3 = 3(x^2 - 1)
  \]
  Set $f'(x) = 0$:
  \[
    3(x^2 - 1) = 0 \implies x = \pm 1
  \]
  Critical points are at $x = -1$ and $x = 1$. To determine nature:
  \[
    f''(x) = 6x
  \]
  - At $x = -1$: $f''(-1) = -6 < 0$ (local maximum)
  - At $x = 1$: $f''(1) = 6 > 0$ (local minimum)
\end{solution}

% PROBLEM 2
\begin{problem}
  Prove that $\sqrt{2}$ is irrational.
\end{problem}

\begin{solution}
  Assume for contradiction that $\sqrt{2} \in \Q$. Then we can write:
  \[
    \sqrt{2} = \frac{p}{q}
  \]
  where $p, q \in \Z$ with $\gcd(p,q) = 1$. Squaring both sides:
  \[
    2 = \frac{p^2}{q^2} \implies p^2 = 2q^2
  \]
  Thus $p^2$ is even, so $p$ must be even. Write $p = 2k$:
  \[
    (2k)^2 = 2q^2 \implies 4k^2 = 2q^2 \implies q^2 = 2k^2
  \]
  Hence $q^2$ is even, so $q$ is even. But this contradicts $\gcd(p,q) = 1$.
\end{solution}

% PROBLEM 3
\begin{problem}
  Find the volume of the solid obtained by rotating the region bounded by $y = \sqrt{x}$, $y = 0$, and $x = 4$ about the $x$-axis.
\end{problem}

\begin{solution}
  Using the disk method:
  \[
    V = \pi \int_{0}^{4} (\sqrt{x})^2 \,dx = \pi \int_{0}^{4} x \,dx
  \]
  \[
    = \pi \left[ \frac{x^2}{2} \right]_{0}^{4} = \pi \left( \frac{16}{2} - 0 \right) = 8\pi
  \]
\end{solution}

% END OF DOCUMENT
\end{document}